
%\pagenumbering{arabic}


%\vspace*{4cm}




%\title{\Large Conserving dynamical mean-field approaches\\to strongly correlated systems\\[4cm]\leavevmode\\
%\large Dissertation\\~\\zur Erlangung des Doktorgrades\\~\\des Fachbereichs Physik\\~\\der Universität Hamburg\\\vspace*{3cm}}
    
%\author{\large vorgelegt von\\Friedrich Johannes Krien\\aus Münster in Westfalen\\2018}
%\maketitle
%\newpage
%\leavevmode
%\newpage

\textbf{Zusammenfassung} In der Dissertation stelle ich eine theoretische Studie von ultraschnellen Prozessen in Systemen mit starken elektronischen Korrelationen vor, die durch Wechselstromfelder aus dem Gleichgewicht gebracht werden. 
Es wird besondere Aufmerksamkeit dem Verständnis des Verhaltens von Systemen mit starken elektronischen Korrelationen in externen Feldern zugeteilt, die bereits im Gleichgewicht reichhaltige Physik demonstriert haben.

Dabei verwende ich die Theorie des dynamischen Nichtgleichgewichtsmolekularfeldes (NE-DMFT) um die stark korrelierter Vielkörpersysteme im
externen Feld zeitaufgelöst zu simulieren.

Durch die Einstellung von Frequenz, Intensität, Polarisation und Pulsform des angelegten Feldes in einem breiten Parameterspektrum sowie durch elektronische Korrelationen erhalten wir Zugang zu einer Reihe neuer Phänomene, die in der vorliegenden Arbeit untersucht werden.

Ein solches Phänomen ist die Vorzeichenänderung der effektiven 
Coulomb~-~Wechselwirkung unter dem Einfluss von Halb-, Mono- und Mehrperiodenpulsen, 
die auf ein zweidimensionales Quadratgitter angewendet werden und kann als realistisches Modell für supraleitende Materialien mit hohem-$T_C$ verwendet werden könnten.

Dann verwende ich NE-DMFT, um das Einfangen von Elektronen in einen neuen metastabilen Zustand zu untersuchen und den Übergang der metallischen Phase zum Isolator zu demonstrieren. 
Dieser Übergang wird von einem effektiven Potential angetrieben, das durch die niederfrequente laserinduzierte Vielkörperdynamik erzeugt wird. 
Diese Phase wird als "Kramers-Henneberger-Festkörper" bezeichnet und in Analogie zum "Kramers-Henneberger-Atom", einem gebundenen elektronischen Zustand, der in einem neuen Potenzial vorliegt, welcher durch die kombinierte Wirkung des Laserfeldes und des Atomkerns erzeugt wurde, diskutiert.

Das Anlegen eines hochfrequenten externen elektrischen Feldes an ein dissipationsfreies Ein-Orbital-Hubbard-Modell  führt zu einer Änderung der Bandstruktur und der Impulsverteilung ohne signifikante Übertragung von Teilchen über das Fermi-Niveau. 
Diese elektronische topologische Modifikation führt zu einem feldinduzierten Lifshitz-Übergang.

Um schließlich einen Einblick in die Natur der magnetischer Anregungen in komplexen itineranten Magneten zu erhalten, verwende ich eine realistische zeit-abhängige Lineare-Antwort-Dichtefunktionaltheorie. 
Ich wende diese Theorie zur Analyse der räumlichen und zeitlichen Skalen von Spin-Dichte-Schwankungen in 3\emph{d} Ferromagneten an.


\newpage
~
\newpage

\textbf{Abstract} In the thesis, I present a theoretical study of ultrafast processes in systems with strong electronic correlations, driven out of equilibrium by ac fields.
The main point is understanding the behavior of a system with strong electronic correlations in the presence of external fields, besides, already reach physics demonstrated by such systems even in equilibrium case.

I use the nonequilibrium dynamical mean-field theory (NE-DMFT) as a method for time-resolved simulation of strongly correlated many-body systems driven out of equilibrium by the external field.

Tuning the applied field in a wide range of frequency, intensity, polarization, and pulse shape together with electronic correlations give us access to a number of new phenomena considered in the present thesis.

First such phenomenon is sign change of the effective Coulomb interaction under the influence of half- mono- and multi-cycle pulses adopted to a two-dimensional square lattice, which could be used as a realistic model of high-$T_C$ superconducting materials.

Then I apply NE-DMFT to study the trapping of electrons into a new metastable state, demonstrating the transition of the metallic phase to the insulator. This transition is driven by effective potential, generated by the low-frequency laser-induced many-body dynamics. This phase is called "Kramers-Henneberger solid", and it is discussed in analogy with "Kramers-Henneberger atom", bound electronic states residing in a new potential generated by the combined action of the laser field and 
the atomic core.

Applying a high-frequency external electric field to a one-orbital Hubbard model without dissipation leads to a change in the band structure and momentum distribution without significant transfer of particles above the Fermi level. This electronic topological modification leads to field-induced Lifshitz transition. 

Finally, in order to gain an insight into the nature of magnetic excitations in complex itinerant magnets, I use realistic linear response time-dependent density functional theory. I apply this theory for analysis of spatial and time scales of spin density fluctuations in 3\emph{d} ferromagnets.




%Last part of the tesis containce analyse of spatial and time scales of spin density fluctuations (SDF)

%Spatial and time scales of spin density fluctuations (SDF) were analyzed in 3\emph{d} ferromagnets using \emph{ab initio} linear response calculations of complete wavevector and energy dependence of the dynamic spin susceptibility tensor. We demonstrate that SDF are spread continuously over the entire Brillouin zone and while majority of them reside within the 3\emph{d} bandwidth, a significant amount comes from much higher energies.  The SDF spectrum is shown to have two main constituents: a minor low-energy spin wave contribution and a much larger high-energy component from more localized excitations. Using the fluctuation-dissipation theorem (FDT), the on-site spin correlator (SC) and the related effective fluctuating moment were properly evaluated and their universal dependence on the 3\emph{d} band population is further discussed.






\newpage


\tableofcontents
%   \begin{eqnarray*}
%   \begin{tikzpicture}
%       \draw[thick] (0,0) -- (16,0) -- (16,3.8) -- (0,3.8) -- cycle;
%        \draw (8,3) node {\Large Reading guide:};
%        \draw (8,2) node {\large A shortened but complete overview of the material is marked with \textbf{\textit{\textcolor{green}{Q}}}.};
%        \draw (8,1) node {\large For results related to physics, skipping technicalities, follow \textbf{\textit{\textcolor{red}{P}}}.};
%   \end{tikzpicture}
%   \end{eqnarray*}
%\setcounter{chapter}{-1}



%\chapter{{Reading guide}}
%\section[A shortened version of the material is marked with \textbf{\textit{\textcolor{green}{Q}}}]{\protect\hyperlink{sec:\thechapter\thesection}{A shortened version of the material is marked with \textbf{\textit{\textcolor{green}{Q}}}}}\toclink
%\section[For results related to physics (no technicalities) follow \textbf{\textit{\textcolor{red}{P}}}]{\protect\hyperlink{sec:\thechapter\thesection}{For results related to physics (no technicalities) follow \textbf{\textit{\textcolor{red}{P}}}}}\toclink
%\section[Click on section headers to return to the table of contents]{\protect\hyperlink{sec:\thechapter\thesection}{Click on section headers to return to the table of contents}}\toclink
\mainmatter
\pagestyle{thesis}

%\chapter{\protect\hyperlink{chap:\thechapter}{A case for the solid state}}
%\chapter{{A case for the solid state}}
%\addtocontents{toc}{\protect\hypertarget{chap:\thechapter}{}}
%\section[Foreword]{\protect\hyperlink{sec:\thechapter\thesection}{{Foreword}}}\toclink
%\addtocontents{toc}{\protect\hypertarget{sec:\thesection}{}}

%Tell a story about five nuts ...

%\section[Structure and scope~\greenq\redp]{\protect\hyperlink{sec:\thechapter\thesection}{Structure and scope}}\toclink\rguideqp  %\marginpar{\textbf{\textit{\textcolor{green}{Q}}\textit{\textcolor{red}{P}}}}[.1cm]
%This thesis is divided into five chapters ...

%\newpage
%\chapter{One- and two-particle correlation functions}\label{sec:viewpoint}
%\begin{center}\textit{"Sooft ich meine Tabakspfeife,\\
%Mit gutem Knaster angefüllt,\\
%Zur Lust und Zeitvertreib ergreife,\\
%So gibt sie mir ein Trauerbild.\\
%Und füget diese Lehre bei,\\
%Dass ich derselben ähnlich sei."}
%\end{center}

%\hspace{6cm}- J. S. Bach
%\vspace{1cm}

%\rguideq This chapter introduces the definitions and the notation used in the rest of the text.
%In the discussion of the many-body Hamiltonians the differences between the continuum and the lattice notation are highlighted.

%\section[Hamiltonians, operators, and correlation functions functions]{\protect\hyperlink{sec:\thechapter\thesection}{Hamiltonians, operators, and correlation functions functions}}\toclink
%\label{sec:hamiltonians}
%\textbf{\textit{\textcolor{green}{Q}}}
%In the continuum, one defines the Hamiltonian of a many-electron system as follows,
%\begin{align}
%H&=\sum_\sigma\int d^3r \Psi^\dagger_\sigma(\rv)\left(-\frac{\Delta_{\rv}}{2}+v(\rv)\right)\Psi_\sigma(\rv)\notag\\
%&+\frac{1}{2}\sum_{\sigma\sigma'}\int\int d^3rd^3r' \Psi^\dagger_\sigma(\rv)\Psi^\dagger_{\sigma'}(\rv')V(|\rv-\rv'|)\Psi_{\sigma'}(\rv')\Psi_{\sigma}(\rv),\label{eq:mb_field_op}
%\end{align}
%Here,
%\begin{align}
%    H_0=\sum_\sigma\int d^3r\Psi^\dagger_\sigma(\rv)\left(-\frac{\Delta_\rv}{2}+v(\rv)\right)\Psi_\sigma(\rv),
%\end{align}
%describes the kinetic motion of many non-interacting electrons in a periodic potential of static ions $v(\rv)$.
%Electrons of spin $\sigma$ are created (annihilated) at location $\rv$ by the fermionic field operators $\Psi^{(\dagger)}_\sigma(\rv)$.
%The Coulomb interaction between the electrons is denoted as
%\begin{align} 
% \hint = \frac{1}{2}\sum_{\sigma\sigma'}\int\int d^3rd^3r' \Psi^\dagger_\sigma(\rv)\Psi^\dagger_{\sigma'}(\rv')V(|\rv-\rv'|)\Psi_{\sigma'}(\rv')\Psi_{\sigma}(\rv),\label{hint}
%\end{align}
%where $V(|\rv-\rv'|)=|\rv-\rv'|^{-1}$ is the Coulomb potential.

%\subsection[Charge and spin densities of the continuum]{\protect\hyperlink{ssec:\thechapter\thesubsection}{Charge and spin densities of the continuum}}\toclinksub
%Of interest for microscopic conservation laws is the time-evolution of the electronic charges and spins.
%In the following, $n(\rv)=\sum_\sigma\Psi^\dagger_\sigma(\rv)\Psi_{\sigma}(\rv)$ denotes the charge density operator in real space. More generally, one defines the density operators,
%\begin{align}
%    \varrho^\alpha(\rv)=\sum_{\sigma\sigma'}\Psi^\dagger_\sigma(\rv)s^\alpha_{\sigma\sigma'}\Psi_{\sigma'}(\rv)\label{def:rho_cont},
%\end{align}
%where $s^\alpha_{\sigma\sigma'}$ are the Pauli matrices. With $s^c_{\sigma\sigma'}=\delta_{\sigma\sigma'}$ one recovers the charge density, $\varrho^c(\rv)=n(\rv)$,
%whereas $\alpha=x,y,z$ yield the spin density operators.
%The operators $S^\alpha=\varrho^\alpha/2$ obey the commutation relations $[S^\alpha,S^\beta]=\sum_\gamma\varepsilon_{\alpha\beta\gamma}S^\gamma$,
%and one may extend the definition of the Levi-Civita symbol to $\varepsilon_{\alpha\beta\gamma}=0$ in case one of its labels $\alpha,\beta,\gamma$ is equal to
%the charge flavor $c$.
%This reflects the fact that the charge density commutes with all other densities and itself, $[\varrho^c,\varrho^\alpha]=0$ for each $\alpha$.
%Despite the characteristic commutation relations, the spin density operators $\varrho^{x,y,z}$ are not proper spin operators, for reasons explained below.

%The above notation is most useful in absence of discrete translational symmetry, such as in the physics of liquid Helium~\cite{Vollhardt84}.
%To approach quantum lattice problems, the continuum notation is rarely used anymore,
%since it does not make use of the periodicity of the ionic potential $v(\rv+\Rv)=v(\rv)$.
%This makes the continuum picture unfavorable in numerical approaches, due to the continuous spatial variable $\rv$. 

%A particular advantage of the continuum over the lattice notation is the simplicity of the Coulomb potential, which is an interaction between charge densities.
%Such an interaction gives an energetic incentive to electrons to avoid each other in order to minimize their potential energy.
%However, the Coulomb potential does not contribute to the charge current, which reflects in the commutativity of the Coulomb potential with the charge density, $[n(\rv),\hint]=0$.
%Likewise, the Coulomb interaction does not cause spin currents either,
%\begin{align}
%    [\varrho^\alpha,\hint]=0,
%    \label{eq:coulombcommute}
%\end{align}
%for the charge and spin densities, $\alpha=c,x,y,z$.
%These commutation relations also become intuitively clear,
%considering that the operator combination $\Psi^\dagger_\sigma(\rv)\Psi^\dagger_{\sigma'}(\rv')V(|\rv-\rv'|)\Psi_{\sigma'}(\rv')\Psi_{\sigma}(\rv)$ acting on a
%Fock state $|\Psi\rangle$ annihilates and recreates two electrons at $\rv$ and $\rv'$ with spins $\sigma$ and $\sigma'$.
%Taking the double integral over the real space, this operation thus merely counts the total potential energy of the electronic
%configuration represented by $|\Psi\rangle$, but leaves the configuration itself unchanged.
%This feature of the Coulomb interaction is exclusive to the continuum notation and lost when making use of the discrete translational symmetry of the ionic lattice.

