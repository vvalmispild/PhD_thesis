\documentclass[aps,preprint,twocolumn,10pt]{revtex4}%
\usepackage{amsfonts}
\usepackage{amsmath}
\usepackage{amssymb}
\usepackage{graphicx}%
\setcounter{MaxMatrixCols}{30}
%TCIDATA{OutputFilter=latex2.dll}
%TCIDATA{Version=5.00.0.2552}
%TCIDATA{CSTFile=revtex4.cst}
%TCIDATA{Created=Sunday, August 29, 2004 11:34:01}
%TCIDATA{LastRevised=Saturday, November 06, 2004 15:39:07}
%TCIDATA{<META NAME="GraphicsSave" CONTENT="32">}
%TCIDATA{<META NAME="SaveForMode" CONTENT="1">}
%TCIDATA{<META NAME="DocumentShell" CONTENT="Articles\SW\REVTeX 4">}
%TCIDATA{Language=American English}
\newtheorem{theorem}{Theorem}
\newtheorem{acknowledgement}[theorem]{Acknowledgement}
\newtheorem{algorithm}[theorem]{Algorithm}
\newtheorem{axiom}[theorem]{Axiom}
\newtheorem{claim}[theorem]{Claim}
\newtheorem{conclusion}[theorem]{Conclusion}
\newtheorem{condition}[theorem]{Condition}
\newtheorem{conjecture}[theorem]{Conjecture}
\newtheorem{corollary}[theorem]{Corollary}
\newtheorem{criterion}[theorem]{Criterion}
\newtheorem{definition}[theorem]{Definition}
\newtheorem{example}[theorem]{Example}
\newtheorem{exercise}[theorem]{Exercise}
\newtheorem{lemma}[theorem]{Lemma}
\newtheorem{notation}[theorem]{Notation}
\newtheorem{problem}[theorem]{Problem}
\newtheorem{proposition}[theorem]{Proposition}
\newtheorem{remark}[theorem]{Remark}
\newtheorem{solution}[theorem]{Solution}
\newtheorem{summary}[theorem]{Summary}
\newenvironment{proof}[1][Proof]{\noindent\textbf{#1.} }{\ \rule{0.5em}{0.5em}}
\begin{document}
\preprint{HEP/123-qed}
\title{Time dependent density functional spin dynamics and its application for Fe and Ni}
\author{Vladimir Antropov}
\affiliation{Condensed Matter Physics, Ames Laboratory, Ames, IA, 50011}
\keywords{spin dynamics, band theory}
\pacs{PACS number}

\begin{abstract}
The main equations of time dependent density functional spin dynamics are
analyzed. Their relation with phenomenological spin dynamics is discussed. The
finite temperature calculations in local density approximation are performed
for ferromagnetic Fe and Ni. The strong magnetic short range order has been
found in Ni. The influence of this dynamical order on observed physical
properties of magnets is shortly discussed.

\end{abstract}
\maketitle


\textit{Ab-initio} spin dynamics\cite{SD,REV,SAGA} recently attracted a lot of
attention due to a possibility to describe magnetic properties of real
materials at finite temperatures without adjustable parameters. In this short
communication we describe the main equations of spin dynamics and analyze
their meaning with emphasis on recently discussed in literature issues. We
also present the results of the application of this dynamics for the studies
of finite temperature properties of 3d ferromagnets.

The evolution of the many-electron system in the presence of the external
field is defined in a unique way by the time-dependent one-particle density
matrix $\rho_{\alpha\beta}\left(  \mathbf{r,r,}t\right)  $, where $\alpha
$,$\beta$ are spin indices. Equivalently, one can introduce the charge
$n\left(  \mathbf{r,}t\right)  =Tr\rho\left(  \mathbf{r,r},t\right)
=\sum_{\nu}f\left(  \varepsilon_{\nu}\right)  n_{\nu}\left(  \mathbf{r}%
,t\right)  $ and magnetization $\mathbf{m}$ $\left(  \mathbf{r,}t\right)
=Tr\rho\left(  \mathbf{r,r},t\right)  \mathbf{\sigma=}$ $\sum_{\nu}f\left(
\varepsilon_{\nu}\right)  \mathbf{m}_{\nu}\left(  \mathbf{r},t\right)  $
densities, where $\mathbf{\sigma}$ are Pauli matrices, $f\left(
\varepsilon\right)  $ is the Fermi function and $\nu$ is a one electron state.
Below we will assume a sum over repeated indices. Starting from the
Schroedinger equation for the many-electron system one can formally obtain the
exact set of the equations for these quantities:
\begin{equation}
\overset{\bullet}{\mathbf{m}}(\mathbf{r,}t)=\gamma\mathbf{m}(\mathbf{r,}%
t)\times\mathbf{B}_{ext}(\mathbf{r,}t)+\frac{i}{2}\mathbf{\nabla}_{\mathbf{r}%
}\mathbf{\nabla}_{\mathbf{r}^{\prime}}(\rho_{\alpha\beta}\left(
\mathbf{r,r}^{\prime}\mathbf{,}t\right)  \mathbf{\sigma}_{\beta\alpha
}-c.c.)_{\mathbf{r}^{\prime}=\mathbf{r}},\label{mexac}%
\end{equation}%
\begin{equation}
\overset{\bullet}{n}(\mathbf{r,}t)=\frac{i}{2}\mathbf{\nabla}_{\mathbf{r}%
}\mathbf{\nabla}_{\mathbf{r}^{\prime}}(\rho_{\alpha\alpha}\left(
\mathbf{r,r}^{\prime}\mathbf{,}t\right)  -c.c.)_{\mathbf{r}^{\prime
}=\mathbf{r}}\label{nexac}%
\end{equation}
where $\mathbf{B}_{ext}(\mathbf{r,}t)$ is the external magnetic field. These
equations are not closed since, generally speaking, the quantity
$\mathbf{\nabla}_{\mathbf{r}^{\prime}}\rho_{\alpha\beta}\left(  \mathbf{r,r}%
^{\prime}\mathbf{,}t\right)  _{\mathbf{r}^{\prime}=\mathbf{r}}$ cannot \ be
described directly in terms of $\rho_{\alpha\beta}\left(  \mathbf{r,r,}%
t\right)  $. Introducing the wave functions of Kohn-Sham quasiparticles
$\varphi_{\nu\alpha}$ as the solution of one-particle perturbed Schroedinger
equation we can obtain a set of equations describing charge and spin dynamics
(see Ref.\cite{SD,REV,SAGA} and references therein):
\begin{align}
\overset{\bullet}{\mathbf{m}}(\mathbf{r,}t) &  =\gamma\mathbf{m}%
(\mathbf{r,}t)\times\mathbf{B}_{tot}(\mathbf{r,}t)+\label{mDF}\\
&  \frac{i}{2}\mathbf{\nabla}_{\mathbf{r}}(\varphi_{\nu\alpha}^{\ast}\left(
\mathbf{r,}t\right)  \mathbf{\nabla}_{\mathbf{r}}\varphi_{\nu\beta}\left(
\mathbf{r,}t\right)  \cdot\mathbf{\sigma}_{\beta\alpha}-c.c.),\nonumber
\end{align}%
\begin{equation}
\overset{\bullet}{n}(\mathbf{r,}t)=\frac{i}{2}\mathbf{\nabla}_{\mathbf{r}%
}(\varphi_{\nu\alpha}^{\ast}\left(  \mathbf{r,}t\right)  \mathbf{\nabla
}_{\mathbf{r}}\varphi_{\nu\alpha}\left(  \mathbf{r,}t\right)
-c.c.),\label{nDF}%
\end{equation}
To make the system complete we have to add the equation for the velocity
function%
\begin{equation}
\mathbf{v}\left(  \mathbf{r},t\right)  =\frac{\hbar^{2}}{2mi}\left(
\frac{\varphi_{\nu\alpha}^{\ast}\left(  \mathbf{r},t\right)  \mathbf{\nabla
}\varphi_{\nu\alpha}\left(  \mathbf{r},t\right)  }{n_{\nu}\left(
\mathbf{r},t\right)  }-c.c.\right)  .\label{velocity}%
\end{equation}
Then the closed system of non linear \textit{one electron} equations for the
local functions $n_{\nu}\left(  \mathbf{r},t\right)  $, $\mathbf{m}_{\nu
}\left(  \mathbf{r},t\right)  $ and $\mathbf{v}_{\nu}\left(  \mathbf{r}%
,t\right)  $ can be presented as (indices of initial and one-electron states
are omitted)%
\begin{equation}
\frac{\partial n}{\partial t}+\partial_{k}\left(  v_{k}n\right)
=0\label{denseq}%
\end{equation}%
\begin{equation}
\frac{\partial v_{i}}{\partial t}+v_{k}\partial_{k}v_{i}=-\frac{1}{m}%
\partial_{i}\left(  V_{Coul}+V_{q}\right)  +\frac{1}{m}\frac{\mathbf{m}}%
{n}\cdot\partial_{i}\mathbf{B}_{tot}-\frac{1}{mn}\partial_{k}T_{ki}%
\label{veloceq}%
\end{equation}%
\begin{equation}
\frac{\partial\mathbf{m}}{\partial t}+\partial_{k}\left(  v_{k}\mathbf{m}%
\right)  =\frac{1}{\hbar}\left(  \mathbf{m\times B}_{tot}\right)  +\frac
{\hbar}{2m}\partial_{k}\frac{1}{n}\left(  \mathbf{m}\times\partial
_{k}\mathbf{m}\right)  \label{momeq}%
\end{equation}
where%
\begin{equation}
V_{q}=\frac{\hbar^{2}}{4m}\left(  \frac{1}{2}\left\vert \frac{\mathbf{\nabla
}n}{n}\right\vert ^{2}-\frac{\triangle n}{n}\right)  +V_{exc}%
\label{qpotential}%
\end{equation}
is the potential energy,
\begin{equation}
T_{ki}=T_{ik}=\frac{\hbar^{2}}{4m}n\partial_{i}\left(  \frac{\mathbf{m}}%
{n}\right)  \cdot\partial_{k}\left(  \frac{\mathbf{m}}{n}\right)
\label{stensor}%
\end{equation}
is the `magnetic' stress tensor and $\partial_{k}$ is $\nabla_{k=x,y,z}$.
$V_{exc}$ is the non-magnetic part of the exchange correlation potential.
These equations are not all independent. We have two constraints (in
insulating case) imposed on those variables%
\begin{align}
n^{2} &  =\mathbf{m}\cdot\mathbf{m}\label{constr1}\\
\mathbf{\nabla}\times\mathbf{v} &  \mathbf{=}\frac{\hbar}{2m}\mathbf{\nabla
}\left(  \frac{m_{z}}{n}\right)  \times\frac{m_{y}\mathbf{\nabla}m_{x}%
-m_{x}\mathbf{\nabla}m_{y}}{m_{x}^{2}+m_{y}^{2}}\label{constr2}%
\end{align}
In addition we have the Poisson equation%
\begin{equation}
\triangle V_{Coul}=-4\pi\left\langle n\right\rangle \label{Poisson}%
\end{equation}
and exchange correlation field which in the local spin density approximation
(LSDA) is presented as%
\begin{equation}
\mathbf{B}_{exc}=B_{exc}\frac{\left\langle \mathbf{m}\right\rangle
}{\left\vert \left\langle \mathbf{m}\right\rangle \right\vert }.\label{Stoner}%
\end{equation}
In the equilibrium%
\begin{align}
\mathbf{m}_{\nu} &  =\left(  0,0,\sigma n_{\nu}^{0}\right)  ,\;T_{ik}%
=0,\label{equistatemT}\\
\mathbf{\nabla}\times\mathbf{v}_{\nu} &  =0,\;\mathbf{B}_{tot}=\left(
0,0,B_{ext}^{0}+B_{exc}^{0}\right)  .\label{equistatevB}%
\end{align}


To demonstrate a connection with phenomenological approaches in the magnetism
theory one can transform second term in the right part of Eq.\ref{momeq} which
represent the kinetic energy contribution to the spin dynamics. It is evident
that a part of this contribution has term proportional to $\mathbf{m}_{\nu
}\times\Delta\mathbf{m}_{\nu}$ which is similar to the generic structure of
the torque term in the phenomenological spin dynamics equation for the total
magnetization\cite{SW}%
\begin{equation}
\overset{\bullet}{\mathbf{m}}\sim\mathbf{m}\times\Delta\mathbf{m=}\left(
\sum_{\nu}f\left(  \varepsilon_{\nu}\right)  \mathbf{m}_{\nu}\left(
\mathbf{r},t\right)  \right)  \times\Delta\left(  \sum_{\mu}f\left(
\varepsilon_{\mu}\right)  \mathbf{m}_{\mu}\left(  \mathbf{r},t\right)
\right)  . \label{qqq}%
\end{equation}
However, in our case
\begin{equation}
\overset{\bullet}{\mathbf{m}}\sim\sum_{\nu}f\left(  \varepsilon_{\nu}\right)
\mathbf{m}_{\nu}\left(  \mathbf{r},t\right)  \times\Delta\mathbf{m}_{\nu
}\left(  \mathbf{r},t\right)  . \label{qq}%
\end{equation}
Evidently the phenomenological Eq.\ref{qqq} is written in rigid spin
approximation and includes many off-diagonal terms of $\mathbf{m}_{\nu}%
\times\Delta\mathbf{m}_{\mu}$ type which are absent in the quantum mechanical
Eq.\ref{qq}.

A connection with the relaxation term in the macroscopical equation of motion
is more complicated. A first type of non-adiabatic processes comes from the
fact that the local magnetization is not an independent dynamic variable. In
general, the dynamics of the local electronic velocity and charge densities is
expected to influence pure spin dynamics in the itinerant magnets. Another
source of non-adiabaticity is related to the summation in Eq(\ref{qq}). This
summation over different states (which might have different time dependences)
leads to such effects as decoherence, the appearance of the optical modes in
systems with one atom per cell and so on. While most of the terms in
Eq.\ref{momeq} have one-electron nature, the exchange-correlation term
($V_{exc}$) can not be presented as such and, in general, takes into account
influence of other electrons in the density functional approach. In LSDA\ the
influence of $\mathbf{B}_{exc}$ on the dynamics of the total magnetization is
absent \cite{REV}, however this term must be included when the dynamics of
one-electron magnetization is studied.

In the non-local case $\mathbf{B}_{exc}$ contributes to the total
magnetization dynamics. Under condition of a weak non-locality one can use a
gradient expansion%
\begin{equation}
\mathbf{B}_{exc}\left(  \mathbf{r},\mathbf{r}^{\prime}\right)  =\mathbf{B}%
_{exc}\left(  \mathbf{r},\mathbf{r}\right)  +A\Delta\mathbf{m}\left(
\mathbf{r}\right)  . \label{e1}%
\end{equation}


These non-local corrections ultimately lead to the familiar torque of
$\mathbf{m}\times\Delta\mathbf{m}$ type (similar to Eq.\ref{qqq}).This term
has functional dependence similar to the kinetic energy term but has a
material coefficient and is obtained in the perturbative way. In general, one
can expect its smallness in the case of well defined local moment systems. The
influence of such gradient terms (non-locality of $\mathbf{B}_{exc}$) on
acoustical branches of spin-wave spectra of 3d ferromagnets was recently
studied\cite{SAGA}. It appears that in the adiabatic regime the addition of
spin angular gradient corrections practically does not change the results
obtained in LSDA. These results justify the usage of weakly correlated LSDA in
cases when magnons are well defined. Also, it indicates that the kinetic
energy term (second in right part of Eq.\ref{momeq}) is a \textit{main} source
of spin dynamics, while the exchange correlation field practically
\textit{static} on this scale of frequencies.

The next source of non-adiabaticity is a time dependence of $\mathbf{B}_{exc}%
$. In the first order%
\begin{equation}
\mathbf{B}_{exc}\left(  t\right)  \simeq\mathbf{B}_{exc}\left(  0\right)
+t\frac{\partial}{\partial t}\mathbf{B}_{exc}=B_{exc}\frac{\mathbf{m}%
}{\left\vert \mathbf{m}\right\vert }+C(t)\overset{\bullet}{\mathbf{m}}
\label{e2}%
\end{equation}
and such an expansion naturally leads to the appearance of the famous
relaxation term of Landau-Lifshitz equation%
\begin{equation}
\mathbf{m}\times\mathbf{B}\times\mathbf{m\sim}\overset{\bullet}{\mathbf{m}%
}\times\mathbf{m} \label{e3}%
\end{equation}
This term is not expected to influence the spin wave spectra but will
contribute to the spin wave decay. It can be easily taken into account in
frame of \textit{ab-initio} spin dynamics described above.

The structure (\ref{e3}) is written in non-relativistic case when the total
magnetic moment of the system is conserved. The account of relativistic
effects will ultimately lead to the appearance of the additional terms in
Eq.(\ref{e3}).

Recently Eq.\ref{mDF} was also discussed in Ref.\cite{GY} and different
reasons have been used to eliminate kinetic energy from spin dynamics
equations. It was stated also that spin dynamics in real materials exists only
due to correlations effects. Consequently, it was concluded that spin dynamics
in LSDA is not correct. However, all these conclusions have been obtained
without any physical or mathematical analysis of Eq.\ref{mDF} and should be
ignored. We demonstrated above that the main contribution to the spin dynamics
comes from the kinetic energy which was completely overlooked in Ref.\cite{GY}.

We applied the technique described above for the temperature dependent spin
dynamical simulations of ferromagnetic Fe and Ni in frame of LSDA using linear
muffin tin orbital method. 20-30 \textbf{k}-points in the irreducible part of
the Brillouine zone in typical calculations with 100-120 atoms per cell have
been used. The correct description of possible long range magnetic order
especially at low temperatures requires the usage of many atoms in simulation
cell. To avoid this numerical difficulty we added spin spiral temperature
dependent boundary conditions\cite{SS}. Effectively such combined
real/reciprocal space spin dynamics allows us to reduce the number of real
atoms per cell in simulations (by factor of 30-35 at $T_{c}$/2 in Ni) and make
statistical calculations possible. From a physical point of view the
simultaneous inclusion of the real space short range modes (inside supercell)
and the long wavelength modes (between supercells) allows us to describe the
magnetic excitations developing on different length scales and their
interactions on equal footing.

The obtained values of $T_{c}$ are very reasonable from a point of view of
experiment and theory. While in Fe we obtained a nearly perfect agreement with
the experiment 1070K (1023K), in Ni the obtained number is about 25\% smaller
than the experimental value (470 K versus 623 K). We believe that our number
in Ni is in compliance with LSDA limitations.

The main result of our calculations is a discovery of a very unusual picture
of magnetic short range order (MSRO) at high temperatures in these
ferromagnets. The degree of MSRO was analyzed by calculating the spin-spin
correlation function $S(\mathbf{r},\mathbf{r}^{\prime},T)=\left\langle
\left\langle \mathbf{m}\left(  \mathbf{r}\right)  \mathbf{m}\left(
\mathbf{r}^{\prime}\right)  \right\rangle \right\rangle .$

It was obtained that the average angle between nearest neighbors moments near
$T_{c}$ is 72$^{0}$ in bcc Fe, while the corresponding angle in fcc Ni appears
to be just 24$^{0}$. Such values indicate that in Fe MSRO is relatively small,
while in Ni it is enormous. Further study indicated that in Ni at high
temperature the dynamical ordering of the distorted spin spiral or domain wall
type is realized with a period of about 6 lattice constants. In Fe, however,
nearest neighbor spin fluctuations are more relevant (soft modes) at $T_{c}$
and relatively weak MSRO of superparamagnetism type arises.

The presence of such MSRO at $T_{c}$ indicates that nearest neighbors
interactions in these materials are larger then $T_{c}$. However, the exchange
parameters in these materials are well known (see, for instance,
Ref.\cite{REV}) and the corresponding numbers are somewhat lower then $T_{c}$.
This contradiction can be eliminated using our recent analysis\cite{JMMM}. It
has been shown that the usage of the exact adiabatic definition of $J_{ij}$
$\sim\left(  \chi\right)  _{ij}^{-1}$(instead of commonly accepted long
wavelength approximation(LWA)) leads to a very different result. First of all,
the traditional LWA is suitable for such a 'localized' system as Fe, and the
corresponding modification of the nearest neighbor $J_{01}$ is relatively
small ($\sim$15\%). Ferromagnetic Ni represents a rather itinerant system and
any local approach (LWA in particular) might produce a large error. In our
case, a large increase in $J_{01}$ for Ni ($\sim$350\% of LWA $J_{01}$) also
support strong MSRO and makes the traditional mean field approach meaningless
($T_{c}\sim$1000K). Correspondingly, LWA can not be used in the itinerant or
any other systems with strong MSRO.

The magnetic itineracy strongly affects the short-ranged part of
$S(\mathbf{r}-\mathbf{r}^{\prime},T)$, which, in turn, influences the
effective range a fixed amount of itinerant electrons with quantum number $l$
and spin $\sigma$ propagates at distance $L$ from a point $\mathbf{r}$ during
thermal fluctuation time $\tau\sim1/T_{c}$%
\begin{equation}
L_{l}^{\sigma}\left(  \mathbf{r}\right)  \sim v_{l}^{\sigma}\left(
\mathbf{r}\right)  \tau S\left(  L_{l}^{\sigma},T_{c}\right)  , \label{a33}%
\end{equation}
where $v_{l}^{\sigma}\left(  \mathbf{r}\right)  $ is the electronic velocity
for given state. The quantity $L_{l}^{\sigma}$ characterizes the degree of
itineracy of the different electrons in magnets.

The discovered MSRO also lead to a new interpretation of the results for the
high temperature susceptibility. The MSRO increase of the effective magnetic
moment in Curie Weiss law is directly related to a clustering (like
superparamagnetism type in Fe) when the effective moment of a cluster is
several times larger then the original atomic moment. The effective moment in
terms of atomic moments is much larger in Ni then in Fe in compliance with
more extended MSRO and experimental data.

Such strong MSRO also allow us to resolve a very old issue of the importance
of quantum corrections for $T_{c}$ and high-temperature susceptibility
calculations. These corrections are traditionally small in the localized
systems (4f magnets), but always considered to be significant in the itinerant
systems. Discovered in Ni MSRO also strongly suppress quantum corrections in
the case of itinerant magnets by increasing the effective moment. We will
describe all these effects in details in our extended publications.

This work was carried out at the Ames Laboratory, which is operated for the
U.S.Department of Energy by Iowa State University under Contract No.
W-7405-82. This work was supported by the Director for Energy Research, Office
of Basic Energy Sciences of the U.S.Department of Energy.

\begin{thebibliography}{9}                                                                                                %


\bibitem {SD}V. P. Antropov, M. I. Katsnelson, M. van Schilfgaarde and B. N.
Harmon, Phys. Rev. Lett. \textbf{75}, 729 (1995); V. P. Antropov. J. Appl.
Phys. \textbf{79}, 5409 (1996); V. P. Antropov, M. I. Katsnelson, M. van
Schilfgaarde, B. N. Harmon, and D. Kusnezov, Phys. Rev. B \textbf{54}, 1019
(1996); V. P. Antropov. in Computer Simulation Studies in Condensed-Matter
Physics XIII. ed. by D. P. Landau. v.\textbf{86}. (Berlin,
Heidenberg,Springer-Verlag, 2001) p.7

\bibitem {REV}V. P. Antropov, B. N. Harmon and A. V. Smirnov. J. Magn. Magn.
Mater. \textbf{200}, 148 (1999).

\bibitem {SAGA}M. I. Katsnelson and V. P. Antropov. Phys. Rev. \textbf{67},
140406 (2003).

\bibitem {GY}J. Kubler. $Theory$ $of$ $Itinerant$ $Electron$ $Magnetism$.
Oxford University Press, Oxford, (2000); K. Capelle and B. L. Gyorffy. Phys.
Rev. Lett. \textbf{87}, 206403 (2001); K. Capelle and B. L. Gyorffy. Europhys.
Lett. \textbf{61}, 354 (2003).

\bibitem {SW}A. I. Akhiezer, V. G. Bar'yakhtar, S. V. Peletminskii. $Spin$
$Waves$ (North-Holland Series in Low Temperature Physics, Vol. 1). 372 pp. (1968).

\bibitem {SS}L. M. Sandratskii, P. G. Guletskii, J. Phys.F \textbf{16}, L43
(1986); V. P. Antropov and M. I. Katsnelson (unpublished).

\bibitem {JMMM}V. P. Antropov. J. Magn. Magn. Mater. \textbf{262}, L192 (2003).
\end{thebibliography}


\end{document}