\chapter{Introduction}

%%%%%%%%%%%%%%%%%%%%%%%%%%%%%%%%%%%%%%%%%%%%%%%%%

In the last decade growth of experimental interest in the field of ultrafast time-resolved spectroscopies in solids. Where an intense
pump laser pulse is used to drive the system into highly
excited states, while another higher frequency probe pulse is used to observe a temporal evolution of the system.
The "pump-probe" technique has enabled the study of transient and relaxation processes which is crucial to understanding ultrafast phase transitions, dynamics of various excitations and many scattering phenomena. This approach has been used to explore a variety of properties in electron systems and ultracold atoms trapped in an optical lattice.

Our fundamental interest is transient nonequilibrium phenomena in strongly correlated
electron systems particular on high-$T_C$ superconductors.
In equilibrium, strong electronic correlations bring plenty of fascinating phenomena,
such as metal-to-Mott-insulator transitions or high-$T_C$ superconductivity and various magnetic phenomena.
If such interacting many-particle systems are
driven out of equilibrium, one can observe rich unexplored variety of dynamics during and
after the perturbation. This will help us to get new physical insights into the correlated system that cannot be discovered in equilibrium.

The fast developments of experimental techniques motivate the progress of theoretical methods to study correlated fermions out of equilibrium. 
Among the many approaches that have been introduced
to study correlated systems, some have already been extended out of equilibrium.
Most successful of them the Dynamical Mean-Field Theory (DMFT) \citep{RevModPhys.68.13,Kotliar} which provides the exact solution in the limit of infinite coordination. 
DMFT approximates only the spatial correlations in a mean-field manner, but accurately treats local temporal fluctuations that are important for describing
strong-correlation phenomena. 
Using Kadanoff-Baym formalism, a general formulation of the Nonequilibrium Dynamical Mean-Field Theory (NE-DMFT) \citep{PhysRevLett.97.266408} was done to describe temporal evolutions of the system from a thermal initial state.

The goal of this thesis is a theoretical investigation
of interacting many-body systems out of equilibrium by tuning the applied pump field in a wide range of frequency, intensity, polarization and pulse shape using NE-DMFT approach.

The structure of the thesis is as follows. 
In \autoref{chap:Non_mb_th}, a brief methodological overview will be given, with an emphasis on NE-DMFT approach and approximations used in it.

In \autoref{chap:pi_pulse} we investigate changing the sign of the effective Coulomb interaction under the influence of half- mono- and multi-cycle pulses. The system adopted to a two-dimensional square lattice, which could be used as a realistic model of high-$T_c$ superconducting materials.

Than in \autoref{chap:KH_solid} we will apply NE-DMFT to study the trapping of electrons into a new metastable state, demonstrating the transition of the metallic phase to the insulator. This transition is driven by effective potential, generated by the low-frequency laser-induced many-body dynamics. This phase is called "Kramers-Henneberger solid", and it is  discussed in analogy with "Kramers-Henneberger atom", bound electronic states residing in a new potential generated by the combined action of the laser field and the atomic core.

Applying high-frequency external electric field to one-orbital Hubbard model without dissipation leads to a change in the band structure and momentum distribution without significant transfer of particles above the Fermi level. This electronic topological modification leads to field-induced Lifshitz transition and will be discussed in detail in \autoref{chap:FS}. 

In the \autoref{chap:MO_FLEX} we will discuss the importance and different ways of including additional orbital degrees of freedom for an accurate description of the electronic structure of correlated materials.

Finally in order to gain an insight into the nature of magnetic excitations in complex itinerant magnets, in the \autoref{chap:TDDFT} we will discuss a brief methodological overview to realistic linear response in the framework of time-dependent density functional theory. We also apply this theory for analysis of spatial and time scales of spin density fluctuations in 3\emph{d} ferromagnets.

%%%%%%%%%%%%%%%%%%%%%%%%%%%%%%%%%%%%%%%%%%%%%%%%%%






\FloatBarrier
