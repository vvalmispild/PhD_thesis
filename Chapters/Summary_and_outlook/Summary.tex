\chapter{Summary and outlook}
The mainline of this thesis is the investigation of correlated electron systems driven out of equilibrium by ac-electric fields using the NE-DMFT approach. Tuning the external electric field in a wide range of frequency, intensity, polarization and pulse shape together with electronic correlations gives us access to a number of new phenomena. 

Also, we discussed the nature of magnetic excitations in complex itinerant magnets using time-dependent density functional theory.

Thus, we summarize the results below:

\vspace*{0.2cm}
$\bullet$ \textbf{Dynamical band flipping with repulsion-to-attraction\: transition in correlated electron systems}

We calculate the time evolution of the isolated Hubbard model on 2D square lattice driven by the ac field using the time-dependent IPT. 
In \autoref{chap:pi_pulse} we show specific conditions for diagonal and circular ac field polarisation wich flips the band structure. This leads to a negative effective temperature and inverted momentum distribution. These results coincide with previous investigations in hypercubic lattice \citep{PhysRevB.85.155124}. 
As a result, the interelectron interaction is effectively converted from repulsive to attractive. This has been numerically confirmed by an increase of the double occupancy beyond the noninteracting value or total energy above its origin.


\vspace*{0.2cm}
$\bullet$ \textbf{Strongly correlated Kramers-Henneberger solid}

In \autoref{chap:KH_solid} we have brought the strong-field concepts developed for atomic systems in the context of strongly correlated
solids. Altering the effective potential for the electron motion with intense pulse light we convert the system from a metallic state to the state
with a Mott gap using the IPT approach. The dynamics in time-domain is resolved via harmonic generation spectroscopy, which encodes the formation of the Mott gap, 
excitation dynamics across it, and the establishment of the insulating state. 
Our findings demonstrate the possibility of manipulating phases 
of correlated systems with strong, non-resonant  
fields in a manner that is extremely robust with respect to the specific
frequency of the driving field, with the time-domain mechanisms opening a new regime of "beyond-Floquet" engineering of strongly correlated systems.

\vspace*{0.2cm}
$\bullet$ \textbf{Nonequilibrium-induced Lifshitz transitions}

In \autoref{chap:FS}, we have introduced a time-dependent field-induced topological transition of the Fermi surface for materials with strong electronic correlations. To do so we apply time-dependent IPT to one-orbital Hubbard model taking into account the nearest and next neighbor's hoppings driven by high-frequency ac field.

As well we discovered the transient increase of the van Hove singularity and in presents of next neighbor's hopping appearing dynamical repulsion of the van Hove singularity from the lower Hubbard band.

\vspace*{0.2cm}
$\bullet$ \textbf{Multi-orbital extension of FLEX self-energy}

We have discussed in the \autoref{chap:MO_FLEX} different implementation schemes of orbital degrees of freedom to the NE-DMFT+FLEX approach. Using the multi-orbital extension of NE-DMFT gives us to rise to rich and more complex physics that cannot be assigned to a single-band representation. 
We compared two-orbital FLEX self-energy approximation in the framework of DMFT with another weak-coupling impurity solvers in equilibrium. Thus FLEX gives better agreement with QMC result, in contrast to a one-orbital case where SOPT the best for half-filling \citep{PhysRevB.91.235114}.

\vspace*{0.2cm}
$\bullet$ \textbf{Spin-density fluctuations in 3d ferromagnetic metals}

Finally to gain an insight into the nature of magnetic excitations in complex itinerant magnets (Fe, Co, Ni) in the \autoref{chap:TDDFT} we have applied linear response in the framework of time-dependent density functional theory. SDF in 3\emph{d} ferromagnetic metals were analyzed. The accuracy of the results was tested by applying two independent calculation methods and establish that the sum rule for the local moment is satisfied both for bare and enhanced susceptibilities. We demonstrated that the SDF are spread continuously over the entire Brillouin zone.

%Since the majority of excitations lie at energies much higher than those accessible by inelastic neutron scattering measurements, different experimental techniques, like spin-polarized high-energy spectroscopies, are required to probe the full SDF spectrum.


\vspace*{0.2cm}


In pump-probe spectroscopy experiments, one usually uses a pulsed pump light with a finite duration. In this work, we carry out calculations for the one-orbital Hubbard model taking into account the pulsed form of the ac field. 
Generally in strongly correlated materials, several orbitals are falling into the low-energy region around the Fermi level. A description of these materials requires an extension of the Hubbard model to the multi-orbital one.  
Thus we believe the development of real-time multi-orbital DMFT+FLEX with avoiding degenerate orbitals and density-density type interaction which was studied in equilibrium \citep{PhysRevB.57.6884,PhysRevB.72.115106} will significantly extend the field of applicability of the model. 



\FloatBarrier